\chapter{Executive Summary}

Executive Summary soll als Zusammenfassung Interesse wecken und enthält einen kurzen Abriss aller wichtigen Aspekte des Businessplans.
\begin{itemize}
\item eine kurze Erläuterung des Produkts / der Dienstleistung
\item der resultierende, relevante Kundennutzen und Wettbewerbsvorteil
\item das zugrunde liegende Geschäftsmodell
\item die Kompetenz des Managements
\item der Investitions- und Finanzbedarf
\item mögliche Umsätze
\item eine Andeutung der Strategie zur Unternehmenszielerreichung sollte ebenfalls nicht fehlen
\item ...
\end{itemize}

\section{Sektion}

\subsection{Untersektion}

\subsubsection{Unteruntersektion}

\section{Sektion}


**Finanzierung**: Die monatlichen Kosten werden selbst durch eine pessimistische Einkommensprognose in Deutschland gedeckt. Die anfänglichen einmaligen Kosten werden von den Gründungsmitgliedern eingebracht. Das Fremdkapital vor allem zur Finanzierung weiterer Mitarbeiter und des Marketings soll von Investoren eingebracht werden.
**Geschäftsplanung und Organisation**: Die Funktionsweise der App ermöglicht es unserem Unternehmen, wertvollen Content in hohem Volumen quasi kostenfrei zu erhalten. Alle benötigten technischen Kompetenzen zur Entwicklung der App werden vom Team bereits gestellt. Weitere notwendige Rollen können aufgrund der individuellen Erfahrungen der Teammitglieder ebenfalls gut verteilt werden.
Die Preisgestaltung sieht niedrige Preise für In-App-Käufe vor, welche gemeinsam mit den eigentlichen Alleinstellungsmerkmalen der App weltweit eine breite UserGemeinde aufbauen sollen. Die User-Gemeinde soll sich ausgehend von Deutschland erst in Europa, dann in den USA und anschliessend auch in östlicheren Regionen verbreiten. Hierfür wird entsprechend dieser Reihenfolge auf gezieltes Online-Marketing und auf Mund-zu-Mund-Propaganda der Foto-Community gesetzt.
