\chapter{Executive Summary}

Der vorliegende Business-Plan stellt eine Geschäftsidee der Mobile-App \textit{Imagical} vor.
\textit{Imagical} ist eine Fotosharing-App für iOS- und Android-Smartphones, mit der Benutzer magische Momente in Bildern festhalten und diese der Community zur Verfügung stellen können. 
Die App überzeugt durch mystisches Design und freischaltbare Features.
Die vier Informatikstudenten Pavan Singh, Sergej Birklin, Nico Wonneberger und Markus Heilig realisieren somit die auf dem App-Markt fehlende Lifestyle-Anwendung, welche die beliebten App-Genres ``Foto'' und ``Spiel'' in einer so nie dagewesenen App vereint.

Um genügend Benutzer auf die Plattform zu locken soll die zielgruppengerechte Vermarktung in erster Linie durch soziale Netzwerke, Suchmaschinenwerbung sowie Mund-zu-Mund-Propaganda erfolgen.

Mithilfe der vereinfachten Stärken-Schwächen-Analyse konnte das grobe Marktpotenzial beziffert werden und zeigt welche Möglichkeiten der Markt und Wettbebwerb bietet. Ferner wurde die Geschäftsidee auch aus einer kritischen Perspektive betrachtet, sodass die tatsächlichen Chancen und Risiken des Vorhabens verdeutlicht werden konnten.


Eine Planung der Geschäfts- und Entwicklungsaktivitäten ergab, dass eine erste Version der App bereits ende November in den App-Stores angeboten werden kann.

Eine vorgenommene Finanzplanung lässt erkennen, dass die monatlichen Kosten selbst durch eine pessimistische Einkommensprognose in Deutschland gedeckt sind. Die anfänglichen einmaligen Kosten werden von den Gründungsmitgliedern eingebracht. Das Fremdkapital vor allem zur Finanzierung weiterer Mitarbeiter und des Marketings soll von Investoren eingebracht werden.
