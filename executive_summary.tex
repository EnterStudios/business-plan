\chapter{Executive Summary}

Der vorliegende Business-Plan stellt eine Geschäftsidee der Mobile-App \textit{Imagical} vor.
\textit{Imagical} ist eine Fotosharing-App für iOS- und Android-Smartphones, mit der Benutzer magische Momente in Bildern festhalten und diese der Community zur Verfügung stellen können. 
Die App überzeugt durch mystisches Design und freischaltbare Features.
Die vier Informatikstudenten Pavan Singh, Sergej Birklin, Nico Wonneberger und Markus Heilig realisieren somit die auf dem App-Markt fehlende Lifestyle-Anwendung, welche die beliebten App-Genres ``Foto'' und ``Spiel'' in einer so nie dagewesenen App vereint.

Um genügend Benutzer auf die Plattform zu locken soll die zielgruppengerechte Vermarktung in erster Linie durch soziale Netzwerke, Suchmaschinenwerbung sowie Mund-zu-Mund-Propaganda erfolgen. 

Eine Planung der Geschäfts- und Entwicklungsaktivitäten ergab, dass eine erste Version der App bereits ende November in den App-Stores angeboten werden kann.

Eine vorgenommene Finanzplanung lässt erkennen, dass die monatlichen Kosten selbst durch eine pessimistische Einkommensprognose in Deutschland gedeckt sind. Die anfänglichen einmaligen Kosten werden von den Gründungsmitgliedern eingebracht. Das Fremdkapital vor allem zur Finanzierung weiterer Mitarbeiter und des Marketings soll von Investoren eingebracht werden.

\mbox{} \\ -------------------------------
Stichpunkte von Pavan:
Der vorliegende Business-Plan stellt eine Geschäftsidee der Mobile-App \textit{Imagical} vor.
\textit{Imagical} ist eine Fotosharing-App, mit der Benutzer magische Momente in Bildern festhalten und diese der Community zur Verfügung stellen können. 
Die App überzeugt durch mystisches Design und freischaltbare Features.
Die vier Informatikstudenten Pavan Singh, Sergej Birklin, Nico Wonneberger und Markus Heilig realisieren somit die auf dem App-Markt fehlende Lifestyle-Anwendung, welche die beliebten App-Genres ``Foto'' und ``Spiel'' in einer so nie dagewesenen App vereint.
>>>>>>> 602e3d6f408cd34a414530ba7e416e8073326894

**Finanzierung**: Die monatlichen Kosten werden selbst durch eine pessimistische Einkommensprognose in Deutschland gedeckt. Die anfänglichen einmaligen Kosten werden von den Gründungsmitgliedern eingebracht. Das Fremdkapital vor allem zur Finanzierung weiterer Mitarbeiter und des Marketings soll von Investoren eingebracht werden.
**Geschäftsplanung und Organisation**: Die Funktionsweise der App ermöglicht es unserem Unternehmen, wertvollen Content in hohem Volumen quasi kostenfrei zu erhalten. Alle benötigten technischen Kompetenzen zur Entwicklung der App werden vom Team bereits gestellt. Weitere notwendige Rollen können aufgrund der individuellen Erfahrungen der Teammitglieder ebenfalls gut verteilt werden. Da wir ein digitales Produkt anbieten, können wir standortunabhängig agieren, was weitere finanzielle Vorteile mit sich bringt.
Die Preisgestaltung sieht niedrige Preise für In-App-Käufe vor, welche gemeinsam mit den eigentlichen Alleinstellungsmerkmalen der App weltweit eine breite User-Gemeinde aufbauen sollen. Die User-Gemeinde soll sich ausgehend von Deutschland erst in Europa, dann in den USA und anschliessend auch in östlicheren Regionen verbreiten. Hierfür wird entsprechend dieser Reihenfolge auf gezieltes Online-Marketing und auf Mund-zu-Mund-Propaganda der Foto-Community gesetzt.
