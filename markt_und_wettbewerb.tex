\chapter{Markt und Wettbewerb}
Laut einer aktuellen Analyse von \textit{Statista} benutzen in Deutschland ca. 50~Millionen Menschen ein Smartphone. Weltweit sind es sogar über 3.5~Milliarden Menschen und rein theoretisch entspricht das der Anzahl möglicher Kunden. Natürlich ist das reale Marktpotenzial deutlich geringer, da zum einen Smartphonebenutzer unterschiedliche Interessen haben und sich möglicherweise Applikationen wie \textit{Imagical} nicht herunterladen. Auf der anderen Seite gibt es Anbieter, die bereits in diesem Markt aufgestellt sind und konkurrieren. Um sich einen besseren Überblick des Marktpotenzials zu verschaffen, soll dieses Kapitel das Marktsegment sowie dessen Konkurrenten genauer analysieren.

Die Geschäftsidee fokussiert sich nicht auf eine bestimmte Altersgruppe, da heutzutage Menschen über alle Altersgrenzen hinweg gerne Fotos machen und diese teilen. Nichtsdestotrotz liegt das Augenmerk aber auf Benutzer zwischen 15 und 35 Jahren. Diese Benutzer bilden die wichtigste Zielendgruppe und erlauben eine genauere Marktsegmentierung. Dazu gehört z.B., dass sowohl weibliche, als auch männliche Kunden angesprochen werden sollen. Ferner soll die Applikation nicht nur in Europa und Amerika vermarktet werden, sondern auch in Asien. Ein Grund hierfür ist, eine maximale Interessensgruppe aufzubauen, da das Teilen von privaten Momenten auch in dieser Benutzergruppe nicht durchwegs Zustimmung findet.

Um das Marktpotenzial herzuleiten muss zunächst die Größe der Zielgruppe bestimmt werden. Eine weitere Studie von \textit{Statista} besagt, dass diese Gruppe einem Anteil von ca. 45~\% entspricht, sodass aktuell 1.58~Milliarden potenzielle Benutzer in Frage kommen. Als logische Annahme soll gesagt werden, dass ein Benutzer täglich durchschnittlich 1~Cent an Einnahmen wie z.B. durch In-App Käufe oder Werbung einbringt. Durch diese Berechnung gelangt man auf ein jährliches Gesamtpotenzial von 5.75 Milliarden~€.

Nachdem nun das potenzielle Marktvolumen beziffert wurde, soll auch ein Blick auf die aktuellen Wettbewerber geworfen werden. Für die Darstellung der Wettbewerber soll die vereinfachte Stärken-Schwächen-Analyse (Tab. 3.1) herangezogen werden.

Die vereinfachte Stärken-Schwächen-Analyse zeigt, dass der Markt noch Potenzial bietet, sofern man natürlich die Schwächen der Konkurrenz ausnutzt und diese in seine eigenen Stärken umwandelt.

(Nützliche Quelle: http://www.statisticbrain.com/in-app-purchase-revenue-statistics/)

\begin{center}
	\begin{table}[htbp!]
	\centering
		\begin{tabular}{| M{2cm} | M{5cm} | M{5cm} | M{3cm} |}
		\hline
			\textbf{ } & \textbf{Stärken} & \textbf{Schwächen} & \textbf{Selbstvergleich} \\ \hline
			Instagram 
			& \begin{itemize}
				\item[]
				\item viele Benutzer
				\item Verbindung anderer sozialer Netzwerke
			\end{itemize}
			& \begin{itemize}
				\item nur für iOS \& Android
				\item Konkurrenten senken direkt Marktanteil
				\item viele qualitativ schlechte Bilder
			\end{itemize} 
			& -+
			\\ \hline
			
			Twitter 
			& \begin{itemize}
				\item viele Benutzer
				\item im Business Bereich beliebt
				\item sehr lebendig
			\end{itemize}
			& \begin{itemize}
				\item beschränkte Zeichen pro Nachricht
				\item kein solides Ertragsmodel
			\end{itemize} 
			& +
			\\ \hline
			
			Pinterest 
			& \begin{itemize}
				\item einfache Navigation
				\item qualitativ hochwertige Bilder
			\end{itemize}
			& \begin{itemize}
				\item viel Spam
				\item Technik nicht ausgereift
			\end{itemize} 
			& ++
			\\ \hline
		\end{tabular}
		\caption{Vereinfachte Stärken-Schwächen-Analyse}
		\label{table:simpleSwot}
	\end{table}
\end{center}
