\chapter{Markt und Wettbewerb}
Laut einer aktuellen Analyse von \textit{Statista} benutzen in Deutschland ca. 50~Millionen Menschen ein Smartphone. Weltweit sind es sogar über 3.5~Milliarden Menschen und rein theoretisch entspricht das der Anzahl möglicher Kunden. Natürlich ist das reale Marktpotenzial deutlich geringer, da zum einen Smartphonebenutzer unterschiedliche Interessen haben und sich möglicherweise Applikationen wie \textit{iMagical} nicht herunterladen. Auf der anderen Seite gibt es Anbieter, die bereits in diesem Markt aufgestellt sind und konkurrieren. Um sich einen besseren Überblick des Marktpotenzials zu verschaffen, soll dieses Kapitel das Marktsegment, sowie dessen Konkurrenten genauer analysieren.

Die Geschäftsidee fokusiert sich nicht auf eine bestimmte Altersgruppe, da heutzutage über Altersgrenzen hinweg Menschen gerne Fotos machen und diese teilen. Nichtsdestotrotz liegt aber das Augenmerk auf Benutzer zwischen 15 und 35 Jahren. Diese Benutzer bilden die wichtigste Zielendgruppe und erlauben eine genauere Marktsegmentierung. Dazu gehört z.B., dass sowohl weibliche, als auch männliche Kunden angesprochen werden sollen. Ferner soll die Applikation nicht nur in Europa und Amerika vermarktet werden, sondern auch in Asien. Ein Grund hierfür ist eine maximale Interessensgruppe aufzubauen, da das Teilen von privaten Momenten auch in dieser Benutzergruppe nicht durchwegs Zustimmung findet.

Um das Marktpotenzial herzuleiten muss zunächst die Größe der Zielendgruppe bestimmt werden. Eine weitere Studie von \textit{Statista} besagt, dass diese Gruppe einem Anteil von ca. 45~\% entspricht, sodass aktuell 1.58~Milliarden potenzielle Benutzer in Frage kommen. Als logische Annahme solll gesagt werden, dass ein Benutzer täglich durchschnittlich 1~Cent an Einnahmen wie z.B. durch In-App Käufe oder Werbung einbringt. Durch diese Berechnung gelangt man auf ein jährliches Gesamtpotenzial von 5.75 Milliarden~€.