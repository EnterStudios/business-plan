\chapter{Unternehmerteam, Management und Personal}


\section{Das Team}

Das Team von \textit{iMagical} besteht aus den vier Gründern Pavan Singh, Sergej Birklin, Nico Wonneberger und Markus Heilig.
Die Gründer kennen sich durch das gemeinsam absolvierte Studium der Informatik nun schon seit über drei Jahren.
Dieses Studium half den Gründern nicht nur der Vertiefung der Kenntnisse im Bereich Software Engineering, sondern schweißte das Team bereits zu sehr früher Zeit eng zusammen, sodass ein unerlässliches, gegenseitiges Vertrauen entstand.
Zudem können die Gründer durch gemeinsam abgeschlossene Teamarbeiten während ihres Studiums auf gemeinsame Erfolge zurückblicken.
Nennenswert ist hierbei die im Rahmen eines Studienprojektes entwickelten App \textit{Locator}, die bereits eine Woche nach Release mehrere hundert Mitglieder zählte.

Mit der Business-Idee \textit{iMagical} will das Team nun den nächsten großen Schritt machen und so die Vision des eigenen Unternehmens wahr werden lassen.

\section{Die Gründer}

\paragraph{Pavan Singh}
\begin{itemize}
\item 2013 - 2016 Studium Angewandte Informatik (Software Engineering)
\item Erfahrungen in der Unternehmensgründung
\end{itemize}

\todo{Hier bitte euer Profil noch etwas pushen! \ldots}

\paragraph{Sergej Birklin}
\begin{itemize}
\item 2013 - 2016 Studium Angewandte Informatik (Software Engineering)
\item Erfahrungen in der Entwicklung von iOS-Apps
\end{itemize}

\paragraph{Nico Wonneberger}
\begin{itemize}
\item 2013 - 2016 Studium Angewandte Informatik (Software Engineering)
\item Erfahrungen in der Entwicklung Android-Apps und Grafik-Design
\end{itemize}

\paragraph{Markus Heilig}
\begin{itemize}
\item 2010 - 2013 Ausbildung als Fachinformatiker (Anwendungsentwicklung)
\item 2013 - 2016 Studium Angewandte Informatik (Software Engineering)
\item Erfahrungen in der Programmierung serverseitiger Systeme
\end{itemize}


\section{Personalplanung}

Alle anfallenden Aufgaben sollen weitestgehend gleichmäßig auf die vier Gründer verteilt werden.
Da Pavan Singh bereits auf profunde BWL-Kenntnisse zurückgreifen kann und sich in in früheren Projekten als Organisationstalent beweisen konnte, wird er die Rolle des CEOs bei \textit{iMagical} einnehmen. Durch sein professionelles Auftreten sowie sein rationales Handeln wird er der Rolle als Unternehmensrepräsentant gerecht.

Die Entwicklung der App wird zwischen Sergej Birklin (iOS-App) und Nico Wonneberger (Android-App) aufgeteilt. Markus Heilig wird die Implementierung des Backends vornehmen. Alle drei können in ihren entsprechenden Gebieten durch Praktika, Ausbildung und Nebenjobs große Erfahrungen aufweisen, weshalb zu Beginn der Einsatz weiterer Entwickler nicht vorgesehen ist.

Da alle vier Teammitglieder durch das Studium mit umfangreichem Technik-Wissen ausgestattet sind, sollen technische Entscheidungen gemeinsam diskutiert und getroffen werden. Somit wird auf die explizite Rollenvergabe ``CTO'' vorerst verzichtet. \\

Nico Wonneberger konnte mit seinen Erfahrungen im Bereich Grafik-Design bereits erste Screen-Designs für die App entwerfen.
Da er jedoch auf Dauer für die Entwicklung der Android-App zuständig sein wird, benötigt das Team weitere Unterstützung im Bereich User-Interface-Design und User-Experience. Hierfür soll ein erfahrener Kommunikationsdesigner angestellt werden, der neben dem App-Design auch die Konzeption und Realisierung des Webauftritts übernehmen soll. Als weiteres Aufgabengebiet wird dieser Person die Umsetzung der zuvor vorgestellten Marketing-Kampagnen in sozialen Netzwerken zugeschrieben.

