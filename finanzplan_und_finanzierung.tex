\chapter{Finanzplanung und Finanzierung}

Wie im Abschnitt „Notwendige Ressourcen“ im Kapitel „Geschäftsplanung und Organisation“ zu entnehmen ist, fallen folgende Kosten an: \\

\begin{tabular}{lc}
  \textbf{Kostenart} & \textbf{Kosten} \\
  \hline
  Arbeitsräumen im Gründerlabor unserer Hochschule & 0\euro \\
  Macbooks & 4196{\euro} (einmalig) \\
  private Grundverpflegung am Anfang & 4000\euro \\
  Reisen und Unterkunft (geschäftlich) & 300\euro \\
  Werbematerial, Flyer und Poster & 50\euro \\
  Patente & 0\euro \\
  Gründungskosten UG & 2000{\euro} (einmalig) \\
  Bargeldbestand & 500\euro \\
  Software-Lizenzen & 0\euro \\
  Weiterbildung & 0\euro \\
  Serverkosten & 100\euro \\
  Marketing & 300\euro \\
  Freelancer & 1000\euro \\
  \hline
  \textbf{Einmalkosten} & 6196{\euro} \\
  \textbf{monatliche Kosten} & 6250\euro \\
  \hline  \\
 \end{tabular}

Diese Kosten beziehen sich auf die Anfangsphase. Mit der Weiterentwicklung des Unternehmens fallen neue bzw. andere Kosten an.

Bei diesem Kostensatz lässt die durchaus pessimistische Einschätzung der monatlichen Einnahmen an knapp 17.000{\euro} zum Entschluss kommen, dass das Vorhaben sich finanziell lohnt. Vor allem das starke Potential im Einnahmenbereich - wie im Abschnitt „Erlösstruktur“ dargestellt - lässt auf diese Erkenntnis schließen.
Für den Anfang bringt jedes Teammitglied 2000{\euro} in das Vorhaben (Einlagen der Gründer), sodass die einmaligen Kosten hierfür abgedeckt werden. Für die monatlich anfallenden Kosten wird es allerdings schon schwierig. Jedoch sollte es bei diesen verhältnismäßig kleinen Summen möglich sein, Investoren gegen eine Beteiligung zu finden, die die monatlichen Kosten stemmen. Das miteingehende Mitspracherecht der Investoren wird aufgrund deren Erfahrungen durchaus begrüßt. Dieses Vorhaben verlangt angesichts der Fachkräfte im Team keine extremen finanziellen Aufwendungen. Dennoch ist es ein risikoreiches Vorhaben, für das wir keinen Kredit aufnehmen möchten, sondern das Fremdkapital ausschließlich von Investoren reinbringen lassen möchten. Aufgrund der oben geschilderten Erlös-Prognosen und der Marktanalyse kommen wir zu dem Entschluss, dass das Vorhaben wirtschaftlich als lohnenswert eingestuft werden kann und die Risiken vergleichsweise niedrig ausfallen.

