\chapter{Chancen und Risiken}
Nicht wenige Menschen müssen in den ersten Jahren ihr Unternehmen wieder aufgeben. Dies liegt nicht immer an Wirtschaftskrisen, sondern ist oftmals eine Folge von schlechtem Management und Fehleinschätzungen. Dazu gehört auch, dass man die Chancen und Risiken des Unternehmens rational abwägen kann und passende Entscheidungen trifft. Eine Möglichkeit, Chancen und Risiken in Verbindung mit den eigenen Stärken und Schwächen zu quantifizieren, ist die sogenannte SWOT-Analyse. Im Folgenden soll dies an einem anschaulichen Beispiel verdeutlicht werden.

\begin{center}
	\begin{table}[htbp!]
	\centering
		\begin{tabular}{| M{2cm} | M{6cm} | M{6cm} |}
		\hline
			\textbf{ } & \textbf{Stärken} (Strength) & \textbf{Schwächen} (Weakness) \\ \hline
			\textbf{Chancen} (Opportunities)
			& \begin{itemize}
				\item []
				\item enge Kundenbindung
				\item durch kurze Entscheidungswege Flexibilität erhöhen
			\end{itemize}
			& \begin{itemize}
				\item flexibel u. unabhängig
				\item ehrgeizig u. ambitioniert
			\end{itemize}
			\\ \hline
			
			\textbf{Risiken} (Threats) 
			& \begin{itemize}
				\item relativ hohe Eintrittsbarrieren im Nieschenmarkt
				\item Erweiterung der Technologieplattform
			\end{itemize}
			& \begin{itemize}
				\item Investorensuche zur Eigenkapitalstärkung
				\item Kooperationen (fehlendes Know-How einholen)
			\end{itemize}
			\\ \hline
		\end{tabular}
		\caption{SWOT-Analyse}
		\label{table:swot}
	\end{table}
\end{center}

Die SWOT-Analyse in Tabelle \ref{table:swot} verdeutlicht zum Beispiel, dass das Gründungsteam von \textit{Imagical} sehr flexibel ist. Im Team werden kurze Kommunikationswege benutzt und muss nicht erst einige Stunden oder gar Tage auf eine Antwort warten. Vielleicht könnte  man es als Schwäche ansehen, dass das Team sehr klein und unstrukturiert wirkt. Aber auch hier sollte man die Schwäche ausnutzen und sich als junges und dynamisches Startup Team präsentieren.

Als Risiko kann natürlich gesagt werden, dass es durchaus mächtige Wettbewerber in diesem Markt gibt. Falls man sich jedoch als Nischenprodukt positionieren kann, sinkt der Wettbewerb indirekt und stellt anderen Startups im Nischenmarkt höhere Barrieren auf.