\chapter{Geschäftsmodell und Organisation}


Geschäftsmodell und Organisation

sehr kurze Beschreibung Geschäftsmodell:
Die (weltweiten) User können sich die Fotoapp kostenlos herunterladen und sind angehalten pro Login-Session mindestens ein
besonderes Foto (kostenfrei) hochzuladen. Somit werden die besonderen Fotoinhalte, welche letzten Endes weitere User anlocken sollen, von der App-Community kostenlos frei gesteuert,
ohne das wir für diese wertvollen Inhalte bezahlen. Je mehr likes einzelne Fotos erhalten, umso mehr Punkte erhält der Hochlader.
Diese Punkte kann er für sonst zahlungspflichtige Features eintauschen oder auch als Eigenmarketing seinen Kanal benutzen. Andere User können besondere Inhalte/Features gegen Punkte/Geld freikaufen,
wodurch das Unternehmen sein Geld verdient.

Kernaktivität: Bereitstellung einer Sharing-Plattform auf der ganz besondere Foto-Inhalte von privaten Usern veröffentlicht werden können.


Das Produkt (Fotoapp) wird hochskaliert und weltweit angeboten. Die interessanten Fotoinhalte werden von den User mindestens bei jedem
Login in die App erstellt, auch weitere Fotos können geladen werden -> skaliert sehr hoch. 
•Ressourcen: Beschreiben Sie, welche Ressourcen notwendig sind, damit Ihr Geschäftsmodell funktioniert. 
Diese Ressourcen können folgender Art sein: physisch (z. B. Gebäude, Maschinen, Fahrzeuge), geistiges Eigentum
 (z. B. Patente, Marken, Software, Kundenprofile) oder finanziell (z. B. Cash, Kreditlinie, Investment) sowie 
 Kompetenzen und ``Know-how'' beispielsweise vom Gründerteam und/oder Personal mit spezifischen Qualifikationen.
 
 xxxxxxxxx
 notwendige Ressourcen:
 -physisch:(z. B. Gebäude, Maschinen, Fahrzeuge)
 - geistiges eigentum:z. B. Patente, Marken, Software, Kundenprofile)
  nicht patentierbar, keine besonderen algorithmen, ausgang eu, nicht usa
 -finanziell (z. B. Cash, Kreditlinie, Investment)
 -Kompetenzen und „Know-how“ beispielsweise vom Gründerteam und/oder Personal mit spezifischen Qualifikationen:
 Ressourcen: 
 vorerst stellen die vier Mitglieder des Teams alle benötigten technischen Ressourcen, da alle Informatiker sind:
 -wichtige Fachbereiche: Entwicklung (Backend/Frontend), Deployment, Performance, Testing, Dokumentation:
 
 Entwicklung verteilt sich über alle:
 Markus:Backend,Performance
 Sergej:Backend,Deployment
 Pavan:Frontend,Testautomatisierung
 Nico:Frontend,Dokumentation
 1-2 Designer: Frontend/Gui
 
 Vorkenntnisse Erfahrungen: Alle AIN-Studenten, viel erfahrung aus Projekten
 Markus:Entwickler bei EADS (backend-erfahrung), Informatik Ausbildung
 Sergej:Backend,Deployment (backend und app entwickler-erfahrung), Marketing
 Pavan:Frontend,Testautomatisierung (testautomatisierung und fe erfahrung,startup-erfahrung (pitching,gründung, finanzen, rechtliches, organisation)), Finanzen
 Nico:Frontend,Dokumentation (mediendesign erfahrung)
 1-2 Designer: Frontend/Gui
 
 Organisation: Scrum, rollenverteilung grob (flache hierarchien)
 entsprechende Rollen vorerst nach aussen (zwangs eines zunächst kleinen Teams):
 technischer vorstand nach aussen (CTO):Markus
 geschäftsführend nach aussen: Pavan 
 
 entsprechend parallel zusätzliche rollen:
 Markus:erschliessung zukünftiger technologischer entwicklungen
 Sergej: Marketing und Vertrieb
 Pavan:finanzen, geldbeschaffung, rechtliches
 Nico:Kundenbetreuung
 
 extern:
 -sobald investorengelder da:
 -auslagerung marketing und vertrieb an externe virma
 -auslagerung kundenbetreuung
 -rechtliche fragen für landesspezifische Gesetzgebung (Datenschutz und sonstiges)
 - zusätzliches mentoring durch eigene investoren
 - lokale vertriebsmanager nach etablierung
 
 
 Die Kompetenzen erstrecken sich über die eingesetzte Programmiersprache und Frameworks, Deployment durch Studium und praktische Erfahrung (PSS),
 die Testkompetenzen ebenfalls, Dokumentation der Software und Entwicklungsorganisation (Scrum)
 Zusätzlich ist im Team Teammitglieder Startup-Erfahrung und Pitch-Erfahrung, vorhanden, sodass man über die ein oder anderen Stolpersteine bereits bekannt sind, genauso
 wie die Wichtigkeit des Mentorings, für das die Investoren natürlich gezielt ausgesucht werden sollen (nicht nur finanzielle Hilfe, sondern auch Domänen-Kompetenz und Hilfe bei Markteintrittsbarrieren
 und professionelle Unternehmensentwicklung und Führung)
 Was fehlt, wäre das Geld für Marketing,es soll versucht werden, dieses einzuholen
 -bargeldbestände können aktuell beschränkt werden auf knapp 500-1000\texteuro für Reise und Unterkunft zu Investoren/Messen (initialkosten für investorenanwerbung)
 -weiterbildungskosten (derzeit abgedeckt), spätere weiterbildung nicht auszuschliessen (wirtschaftlich/technisch)
 -materialkosten für werbematerial auf messen oder für investoren
 
 zusatz:
 Unternehmerteam, Management und Personal
-möglcihst kleine Teamgrösse: aktuell vier personen, (ohne designer)
-leute kennen sich schon lange, verstehen sich gut, haben schon gemeinsam projekte betreut
-alle vier sind überzeugt von der idee und möchten diese gemeinsam vorantreiben.
-alle vier arbeiten unabhängig, können hürden bewältigen.
-so überzeugt, dass jeder 2000 € mit einbringt.
 
 Partnerschaften:
 Partnerschaften wird es zuerst mit Investoren zwangsläufig geben.
 Erwartungen an diese: Startup Kompetenz im Medienbereich/Werbebereich.
 Kompetenzen in Markteinführung, Weiterenticklung, marketing und Vertrieb.
 
 technische Partnerschaft: Server von Amazon. 
 Eigenschaften: gute flexible Kostenstruktur, zuverlässige in Europa platzierte 
 Server für besseren Datenschutz.Hierfü werden natürlich Kosten fällig.
 
 Partnerschaften können sich evtl aus den Vertriebswegen ergeben, dass Firmen
 gegen kleinen Anteil zu festgelegten Konditionen /festgelegte Art unser Unternehmen vermarkten
 bzw. aktiv nutzen und somit ihre eigene Community auf unsere App aufmerksam machen.
 
 Möglicherweise Partnerschaften mit besonderen Erlebnisunternehmen (Otto-200E Erlebnisgutscheine), Reiseuntrnehmen,
 mit denen ergeben sich synergien. Das ganze ist allerdings nur ein Ausblick.
 Es wird nicht nur werbung für unsere app gemacht, die ja besondere momente festhält, sondern auch für die unternehmen selbst,
 da diese besondere momente bewerben.
 
 z.b. mit ausrüsterfilmen: wie z.b. canon spiegelreflexkameras oder samsung-hersteller mit ganz besonders guter kamera, die geeignet istu
 um besondere momente festzuhalten auf unserer app
 solche Unternehmen haben ganz andere kernaktivitäten, die sich mit unseren perfekt ergenzen.
 somit keine konkurrierende Interessenverfolgung, denn diese unternehmen konkurrieren z.b.
 auch nicht mit unseren konkurrenten (Instagram, pinterest, etc.)
 
 -make or buy:
 für die kernaktivitäten bestehen derzeit alle benötigten technischen kompetenzen im eigenen team.
 -für das marketing könnte eine marketing-firma engagiert werden, bzw. vertriebsleute eingestellt werden,
 da professioneller marketing/vertrieb nicht zu unseren kernkompetenzen gehört.
 -später supportdienst outsourcen

 kostenstruktur: wichtigste kostentreiber:
 marketing/vertrieb
 die meisten leute haben gute smartphones mit top-ausgestatteten kameras, daher ist das keine sorge
 
 -value proposition/nutzenversprechen: wir ermöglichen das sehen und teilen von ganz besonderen momenten im fotoformat.
 
 •Kundensegmente: Beschreiben Sie, wen genau Sie als Kunden erreichen und bedienen wollen.
 -kundensegment: erstlinig die junge generation: ab 16 bis 35, aber natürlich auszuweiten für alle alterklassen, wobei
 die älteren natürlich nicht so häufig auf smartphones/social networks herumtreiben. 
 
 aus diesem grund wird auch gezielte werbung für diese Zielgruppe gemacht, siehe marketing.
 Die Kunden erstrecken sich über den Globus, allerdings macht es sinn sich auf die Kundschaft entsprechend unserer Marketingstrategie zu konzentrieren,
 sprich:geographischer Aspekt der Kunden: Deutschland, UK, Frankreich, Spanien, wenig gelder in andere europische länder,
 hauptgelder schon eher dann in usa, da in europa über die grossen länder bereits mund zu mund propaganda betrieben wird 
 und man hier an werbegeldern etwas sparen kann für die restlichen verbleibenden länder
 
 -Vertriebs/Kommunikationskanäle:
 Über facebook zielgruppe wählen, dann bezahlen, rest übernimmt facebook.
 Kriterien für facebook-wahle zum einstellen:
 Alter: 16-34
 -hobbys/interessen: reisen, life, happiness, travel, adventure, landesspezifische hashtags love,paris,berlin,etc.
 -hauptsächlich europa, Nordamerika asien  zuerst in dieser reihenfolge.
 damit sollte schon genug marketing geld fürs erste ausgebucht sein.
 je etablierter die app wird (schneeballeffekt), umso besser läuft die mund zu mund propaganda
 umso mehr wird der vertrieb von selbst statt finden. dennoch: wenn der cashflow stärker wird (gewinne nicht unbedingt)
 kann man das geld gleich für weitere werbungen einsetzen, vor allem verstärkt in asien und in den anderen noch nicht erfassten 
 kontinenten/ländern, bevor hier nachahmer aktiv werden. Geschwindigkeit zählt am anfang mehr, dadurch sind langfristige rentabilitäten eher garantiert (schutz vor nachahmer)
 
 
 -spätere vertriebswege über synergetische Partnerschaften:
 (optimalzustand): Iphone-Werbung oder gutschein werbung, die in ihren Werbungen unsere app einblenden
 bzw. die app erwähnen: schöne momente festhalten und teilen (hier wird unsere app kurz eingeblendet)

 -spätere vertriebswege über hotels (hotelketten): hochzeiten bei hilton als besondere momente festhalten.
 werbung für hotelkette, für uns gratis-werbung. Um auf solche unternehmen verstärkt einzugehen und nicht nur weltketten
 sondern auch lokal fest verwurzelte international bekannte niederlassungen/events zu erreichen (oktoberfest, hofbräuhaus in münchen)
 werden hierfür vertriebsmanager aufgestellt, denen im allgemeinen ein eigener distrikt/geographischer Bereich für den vertrieb zugeordnet ist,
 sodass auf lokal verwurzelte unternehmen viel individueller eingegangen werden kann. 
 Das kernkonzept dieses vertriebsweges ist es, dass man nicht unbedingt den Kontakt mit dem Endverbraucher (dem User) sucht, sondern mit Unternehmen,
 die die app für werbezwecke einsetzen könnten und so deren werbung gratis finanzieren und durch den renommierten namen (hilton hotelkette) zum
 ruf dieser app beitragen (und somit dazu beitragen, dass sich unsere app viel schneller zur marke entwickelt, statt eine normale app zu sein)
 erlösstruktur:Folie 172
 
 -Standortwahl
Standort-wahl: zunächst zuhause/keller, relativ unabhängig vom standort, hauptsache zentral in europa gelegen, KN als dreiländerdreieck ideal.
steuerliche vergünstigungen: keine grossen unterschiede in europa
-branchenspezifisches umfeld nicht nötig
-kundennähe nicht nötig, da digitale arbeit
-geeignete Mitarbeiter: universitätsstadt, dreiländerdreieck mit internationalen fachkräften
-staatliche förderungen von land und bund für startups vorhanden
-rohstoffnähe unbedeutend

 xxxxxxxxxxxxxxxxxxxxxxxxxxxxxx
 


Kosten für den Kunden und Preispolitik
-Gestaltung der Konditionenpolitik
1. Preise werden individuell mit Kunden aushandelt:
das wird es nicht geben:
vertrieb über partnerfirmen: die firmen sollen nicht bezahlen, da wir selbst durch deren gratiswerbung profitieren
für die kunden (user) werden die preise gleich sein, features werden für alle den gleichen preis haben (unabhäging von alter/beruf)
individuell: user, die besonders viele likes erhalten oder sehr aktiv sind auf der app, erhalten bonuspunkte und können somit features gratis freischalten (individuelle kostenstruktur)
2.mengenmässig: user können feature-pakete kaufen: fotos ansehen, ohne selbst zu schiessen (1€ pro monat), fotos nicht bewerten müssen bei ansicht (1€ pro monat), featurepaket: kommentare, username, fotofilter (1€ pro monat)
3.räumlich: kostenstruktur ist ähnlich in anderen ländern, in europa und usa gleich (in usa den europreis in dollar), asien: niedrigere gebühren


-Erlösstruktur: Ähnlich wie bei der Darstellung der Kostenstruktur, können Sie basierend auf den Schritten sechs bis acht – und hier insbesondere mit realisierbaren Verkaufstransaktionen – zeigen, wie Sie Geld verdienen.
Deutschland: 82 mio einwohner; laut studie xxxx(statistisches bundesamt, 2014) 17 mio in alter 15-34:
http://www.spiegel.de/politik/deutschland/demografie-deutschland-altert-trotz-zuwanderung-a-1073216.html
wenn von denen auch 10% sich für die app  entscheidet,
und von denen auch nur 1% sich für ein paket entscheidet, hat man monatlich einnahmen von  17 mio * 0,001 = 17 000€ (das sogar nur pessimistische rechnung)

Es ist jedoch bei guter arbeit mit einem faktor von bis zu 10 zu rechnen, womit man die vielen mitarbeiter finanzieren könnte und kredite abbezahlen könnte.

 -> welche rechtsform macht sinn? GbR? GmbH? AG?
 

Finanzplanung und Finanzierung
•Konsolidierung aller zuvor getroffenen Annahmen in einem stimmigen Zahlenwerk, bestehend aus Gewinn-und-Verlust-Rechnung, Liquiditätsplanung, Bilanz und den zugehörigen Detailplanungen.
•Mit der Finanzplanung prüfen Sie die Plausibilität und finanzielle Machbarkeit Ihres Vorhabens.
•Bestimmung der Finanzierungslücke und Möglichkeiten, diese zu schließen.
Folie 184!!! Gesamptlanung
Folie 185!!!
Folie 186!!
Bilanz Folie 187!!!
Folie 188


Finanzplanung und Finanzierung
- Eigenkapitalfinanzierung: Bei einer Eigenkapitalfinanzierung beteiligt sich ein externer Investor am Unternehmen, d. h., er erhält Unternehmensanteile für seine Einlage. Hierdurch hat er die Chance, am Unternehmensgewinn teilzuhaben, muss allerdings auch das Risiko tragen sein, eingesetztes Kapital möglicherweise ganz zu verlieren. Des Weiteren berechtigt es in der Regel, bei Unternehmensentscheidungen zumindest ein Mitspracherecht zu haben. •
•Einlagen der Gründer
•Business Angels
•Venture Capital (VC)
•Börsengang

•Fremdkapitalfinanzierung: Fremdkapital ist Kapital mit Rückzahlungsanspruch, in der Regel in Form von Zinsen und Tilgung. Kredite bzw. Darlehen sind die wichtigsten Formen der Fremdkapitalfinanzierung. Bei der Aufnahme eines Kredits wird Kapital an einen Schuldner ausgeliehen. Kredite unterscheiden sich voneinander durch die Laufzeit und die Konditionen, die v. a. die Verzinsung, die Rückzahlung und die geforderten Sicherheiten betreffen.


Finanzplanung und Finanzierung
•Mezzanine-Finanzierung: Die Mezzanine-Finanzierung stellt eine Mischung aus Eigen- und Fremdkapital dar. Meist handelt es sich dabei um Kredite, deren wesentliche Merkmale sind, dass keine Besicherung erforderlich ist und sie einen Rangrücktritt hinter andere Gläubiger aufweisen. Mezzanine-Kapital wird somit zu wirtschaftlichem Eigenkapital. Im Unterschied zur klassischen langfristigen Kreditfinanzierung besteht ein Rückzahlungsanspruch für den Fall der Insolvenz erst nach den anderen Gläubigern.


Finanzplanung und Finanzierung (PAVAN)
    EINNAHMEN:
    -> Wie teuer ist ein In-App Kauf bei anderen Apps?
    -> Wie viele User brauchen wir also, dass wir unsere Kosten decken?    
    KOSTEN
    -> was sind einmalanschaffungen (gemeinkosten) -> büroeinrichtung, server, pc
    -> was sind laufende kosten (betriebskosten) -> miete, strom, heizung, sw-lizenzen, grunddeckung p.P. 800€-1000€
    -> geld für werbung -> zahlen im internet recherchieren
    -> gründungskosten für unternehmensanmeldung    
    -> wie viel fremdkapital? wir brauchen geld für werbung!
    -> wir haben etwas eigenkapital (jeder hat 2000€ in die firma investiert)






















