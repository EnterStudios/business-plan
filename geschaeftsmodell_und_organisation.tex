\chapter{Geschäftsmodell und Organisation}


\section{Beschreibung des Geschäftsmodells}

User können sich die Fotosharing-App kostenlos herunterladen und sind angehalten, pro Login-Session mindestens ein besonderes Foto (kostenfrei) hochzuladen. Somit werden die besonderen Fotoinhalte, welche letzten Endes weitere User anlocken sollen, von der App-Community kostenlos zur Verfügung gestellt. Auf diese Weise müssen wir selbst keine professionellen Fotografen anstellen. Dennoch erhalten wir sehr wertvollen Content – und das unentgeltlich. Aufgrund der in der Produktbeschreibung erläuterten Funktionsweise der App ist zudem der Vorteil einer hohen Skalierbarkeit im Sinne eines exponentiellen Wachstums des Contents gegeben. Diese Konstellation ist in dieser Form in keiner anderen Foto-App gegeben.

Je mehr Likes einzelne Fotos erhalten, umso mehr Punkte erhält der User, der diese Fotos hochgeladen hat. Diese Punkte kann er für sonst zahlungspflichtige Features eintauschen. Als Nebeneffekt kann er seinen Kanal auch für Eigenmarketing-Zwecke benutzen, was für ihn eine zusätzliche Motivation für guten Content auf unserer App sein ist und weitere Nutzer anlocken soll.

Alternativ können User Features und den Zugang zum Content auch mit Geld freikaufen, wodurch unser Unternehmen seine Einnahmen macht.


\section{Beschreibung der Kernaktivität}
Die Kernaktivität besteht darin, dass eine Sharing-Plattform bereitgestellt wird, auf der ganz besondere Foto-Inhalte von privaten Usern geteilt werden können.



\section{Notwendige Ressourcen}
Für einen einwandfreien technischen Entwicklungsverlauf benötigen wir einen ruhigen Arbeitsraum für unser zunächst vierköpfiges Team. Da wir nur digital in Kontakt mit unseren Usern treten, ist wir für den Anfang kein Präsentationsraum oder sonstiger gewerblicher Raum notwendig, sodass wir uns mit den kostenlosen Arbeitsräumen im Gründerlabor unserer Hochschule zufrieden stellen können (Kosten: 0\euro). 
Weiterhin benötigen wir eine gute technische Ausrüstung, damit es nicht zu unnötigen Verzögerungen bei der technischen Entwicklung kommt. Hierfür sollen insgesamt vier leistungsfähige Macbooks zum Einsatz kommen (einmalige Kosten: 4 * 1.049\euro = 4196\euro).
Für die private Grundverpflegung reicht uns pro Kopf ein monatlicher Betrag von knapp 1000\euro, da wir alle noch Wohnräume zu Studentenpreisen beziehen und begrenzte Ausgaben haben. In diesem Betrag sind auch die Kosten für Monatstickets für den öffentlichen Verkehr in Konstanz enthalten (Kosten: 4 * 1000\euro = 4000\euro).
Es ist davon auszugehen, dass wir monatliche Kosten für Messenbesuche zur Eigenwerbung und zur Findung von Investoren benötigen werden. Hierfür sind hauptsächlich Reisen nach Berlin vorgesehen, welche als Startup-Metropole in Europa gilt. Sollten wir monatliche eine Reise machen und zwei Teammitglieder entsenden, können wir für Reise, Verpflegung und Unterkunft insgesamt 300\euro ansetzen (Kosten: 300\euro).
Für schriftliches Werbematerial (Flyer und Poster) veranlassen wir monatliche Kosten von 50\euro (Kosten: 50\euro).
Software kann in Europa nicht patentiert werden, auch in den USA lassen sich höchstens spezielle Algorithmen schützen. Da wir uns nicht im wissenschaftlichen Bereich bewegen, werden für Patente keine Kosten anfallen (Kosten: 0\euro).
Weitere anfängliche Kosten sind für die Gründung einer UG und die anfallenden Notariatskosten notwendig. Hierfür setzen wir knapp 2000\euro als einmalige Kosten an (Kosten einmalig: 2000\euro).
Für kleinere Ausgaben möchten wir monatlich eine Bargeldmenge von 500\euro bereithalten (Kosten: 500\euro).
Für die Entwicklung kommt Open-Source-Software zum Einsatz, sodass hierfür keine besonderen Rücklagen gemacht werden müssen (Kosten: 0\euro).
Weiterbildungsmöglichkeiten im technischen und wirtschaftlichen Bereich werden im Internet genügend und meistens kostenfrei angeboten  (Kosten: 0\euro).
Die Server beziehen wir flexibel von Amazon Virtual Servers. Die für den Anfang benötigte Leistung dürfte sich in Grenzen halten, wir gehen von Ausgaben von maximal 100\euro pro Monat aus  (Kosten: 100\euro).
Besondere Kostenpositionen ergeben sich aus dem Marketing, wofür wir 10\euro pro Tag für zunächst 50 Tage einplanen. Je nach Verlauf der Entwicklung könnten diese danach deutlich ansteigen und einen Kostenpunkt von mehreren Tausend Euro pro Monat ausmachen. Voraussetzung hierfür wäre, dass wir rasant an Nutzer gewinnen. Dennoch setzen wir die anfänglichen Kosten hierfür auf knapp 300\euro monatlich  (Kosten:300\euro).
Zusätzlich werden wir die Dienste eines Designers benötigen, den wir zunächst auf Freelancer-Basis anstellen möchten. Die Kosten beziffern wir auf 1000\euro monatlich, da die Designer-Tätigkeit keine Vollzeitbeschäftigung sein wird (Kosten: 1000\euro).

Weitere Ressourcen werden nicht benötigt. Vor allem die technischen Ressourcen werden vom Team voll umfänglich gestellt, da es sich bei jedem Mitglied um einen Informatik-Studenten handelt, der die notwendigen Fähigkeiten und Erfahrung einbringt. An dieser Stelle sei auf das Kapitel „Unternehmerteam“ verwiesen, das genauer auf die Fähigkeiten der Teammitglieder und teilweise auch schon auf die Rollenverteilung eingeht.


\section{Organisation}
Auf die Rollenverteilung und die Gründe für die Eignung des Teams wird bereits im Kapitel „Unternehmerteam“ eingegangen. Unsere Organisation basiert auf Scrum, einem Vorgehensmodell des Projekt- und Produktmanagements im Softwarebereich. Auf flache Hierarchien wird großer Wert gelegt. Zusätzlich zu den im Kapitel „Unternehmerteam“ erwähnten Rollen erfolgt zwecks Organisation eine weitere Verteilung von Aufgaben:

Markus: Erschließung zukünftiger technologischer Entwicklungen
Sergej: Marketing und Vertrieb
Pavan: Finanzen, Geldbeschaffung, Rechtliches
Nico: Kundenbetreuung

Diese Aufgabenfelder bedeuten neben der vorhandenen Rollenverteilung eine weitere Belastung, die nicht dauerhaft getragen werden kann – erst recht nicht bei wachsendem Geschäft. Daher soll es einige Veränderungen geben, sobald Investorengelder da sind. Diese Veränderungen umfassen

\begin{itemize}
\item die Auslagerung von Marketing und Vertrieb an eine externe Firma
\item die Auslagerung der Kundenbetreuung an eine externe Firma
\item auch rechtliche Fragen insbesondere für die landesspezifischen Gesetzgebungen für Datenschutz und Sonstiges sollen an Fachleute übergeben werden
\item langfristig werden abhängig vom Unternehmenserfolg lokale Vertriebsmanager eingesetzt werden, die weitere Marketingmöglichkeiten nach geographischer Aufteilung finden sollen
\item auch die Investoren sollen durch ihr Mentoring in Entscheidungsprozessen eingebunden werden
\end{itemize}


\section{Partnerschaften}
Partnerschaften wird es zwangsläufig zuerst mit Investoren geben. An diese stellen wir folgende Erwartungen:
\begin{itemize}
\item Sie sollen Geld für die Grundkosten und vor allem für das Marketing und die Anstellung weiterer Fachkräfte (bspw. Designer) aufbringen
\item Sie sollen uns in der Entscheidungsfindung mithilfe ihrer Erfahrung und ihres Netzwerkes helfen.
\end{itemize}

Weitere Partnerschaften könnten sich möglicherweise aus Vertriebswegen ergeben. Dies ist allerdings eher ein sehr langfristiger Ausblick. Beispielsweise könnte es Partnerschaften mit Erlebnisunternehmen geben, wie Reiseunternehmen oder Unternehmen, die Gutscheine für im Freizeitbereich verkaufen (Skydiving, Rennauto fahren, etc.). Sie alle stehen dafür, dass sie besondere Erlebnisse „verkaufen“ möchten, die wir als besondere Momente für unsere Foto-App auffassen. Hier könnte es Synergien geben: Solche Unternehmen werben mit unserer App für ihre ganz besonderen Momente. Davon profitieren nicht nur die Unternehmen, sondern auch unsere App erreicht einen höheren Bekanntheitsgrad.

Ähnliche Synergien könnten sich beispielsweise mit Smartphone-Hersteller wie Samsung ergeben, die bekanntlich mit hochqualitativen Fotokameras werben. Sollte unsere App in Zukunft eine große Usergemeinde haben, könnten diese Smartphone-Hersteller damit werben, dass sie die ideale Kamera für diese App bieten. Diese Unternehmen betreiben somit Werbung in eigener Sache, von der wir gleichzeitig kostenfrei profitieren, da unsere App mitvermarktet wird. Eine konkurrierende Interessensverfolgung gäbe es in dem Fall nicht. Das ist jedoch ein sehr langfristiger Ausblick.


\section{Make or Buy}
Für die Kernaktivitäten in technischer Hinsicht sind alle Kompetenzen im Team bereits vorhanden. Idealerweise sollten für das Marketing Fachkräfte und Vertriebsleute engagiert werden, da professionelles Marketing nicht zu unseren eigentlichen Kompetenzen gehört. Auch der Support soll mittelfristig ausgelagert werden.


\section{Kostentreiber}
Die wichtigsten Kostentreiber sind weiter oben bereits erwähnt worden. Diese sind hauptsächlich Marketingkosten und Personenkosten (Designer).


\section{Value Proposition/Nutzerversprechen}
Wir ermöglichen das Sehen und Teilen von ganz besonderen Momenten im Fotoformat.


\section{Kundensegmente}
Wie im Abschnitt Markt- und Wettbewerbsanalyse beschrieben, fokussieren wir uns erstlinig auf die Altersklasse 15 bis 35, da die älteren Generationen eher selten auf Smartphones oder Webplattformen zurückgreifen. Entsprechend der Ausführungen im Marketing-Abschnitt sollen zunächst Anstrengungen auf dem europäischen Markt, dann in den USA und schließlich langfristig auch in Asien gemacht werden.


\section{Standortwahl}
Da unser Produkt im digitalen Bereich angesiedelt ist, sind wir relativ unabhängig vom Standort. Aktuell können wir die kostenfreien Räume unserer Hochschule benutzen. Mit Konstanz sind wir ohnehin ideal im Dreiländereck platziert. Bezüglich steuerliche Vergünstigungen gibt es angesichts unserer derzeitigen Situation keine großen Unterschiede zu benachbarten Ländern. Auch ein branchenspezifisches Umfeld ist derzeit nicht notwendig. Eine geographische Kundennähe ist nicht notwendig. Geeignete Mitarbeiter gibt es aufgrund der Universität und der Hochschule in Konstanz genügend. Eine besondere Rohstoffnähe ist auch nicht wichtig für dieses Vorhaben.



\section{Kosten für den Kunden und Preispolitik}
Gestaltung der Konditionenpolitik: Unsere Preise für die In-App-Käufe werden nicht individuell ausgehandelt. Man kann pauschal formulieren, dass die Preise landesspezifisch sein werden und sich bis auf den viel später anvisierten asiatischen Markt nicht groß unterscheiden.
Erwähnenswert in diesem Zusammenhang wäre, dass die Konditionenpolitik insofern „individuell“ wäre, als dass die User sich Features auch anhand von Punkten freischalten lassen können.

Mengenmäßige Preisgestaltung: User können Feature-Pakete kaufen:
\begin{itemize}
\item Fotos ansehen, ohne selbst eins zu schießen :1\euro pro Monat
\item Fotos sehen, ohne diese bewerten zu müssen: 1\euro pro Monat
\item Feature-Paket: Kommentare, Username, Fotofilter: 1\euro pro Monat
\end{itemize}


\section{Erlösstruktur}
Am Beispiel von Deutschland, dem Land, in dem wir die Marketing-Kampagne starten, soll die Erlösstruktur veranschaulicht werden:
Deutschland hat aktuell 82 Mio Einwohner. Laut einer Studie des Statistischen Bundesamt aus 2014 (Quelle: http://www.spiegel.de/politik/deutschland/demografie-deutschland-altert-trotz-zuwanderung-a-1073216.html) sind hiervon 17 Mio Einwohner im Alter zwischen 15 und 34. Wenn sich hiervon 10\% für die App entscheiden, und davon auch nur 1\% sich für ein Kostenpaket entscheiden, hat man monatliche Einnahmen von 17.000\euro, was sogar nur einer pessimistischen Schätzung gleich käme. Abhängig vom Erfolg der App und der damit verbundenen Arbeitsschritte (Marketing, Design, etc.) kann mit einem Faktor im niedrigen zweistelligen Bereich bei den monatlichen Einnahmen in Deutschland gerechnet werden. Da die App aber auch international expandieren soll und sich hier die Marketingkosten anfänglich proportional erhöhen und dann durch  Mund-zu-Mund-Propaganda begrenzt werden, liegt hier nochmal ein potenziell hoher Multiplikationsfaktor bei den Einnahmen bei relativ hierzu sinkenden Ausgaben vor. Mit diesen Einnahmen ließen sich die Investorengelder schnell abbezahlen und laufende Kosten (hauptsächlich Mitarbeiter und Marketing) könnten aus eigener Kraft leicht gestemmt werden.


\section{Rechtsform}
Es macht Sinn, das Unternehmen anfangs als UG zu gründen und diese ab einem Kapital von 25000\euro in eine GmbH umwandeln zu lassen. Somit wären wir privat rechtlich geschützt und könnten auch die Anteilsregelung mit Investoren am besten realisieren.

