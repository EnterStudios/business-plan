\chapter{Marketing und Vertrieb}

Das Ziel unseres Marketings soll sein, durch die Anstrengungen des Chief Marketing Officers bis in einem Jahr 10.000 aktive User zu generieren.
Die App hat auf den ersten Blick nur Nachteile für den User. Der User muss Bilder machen um die App zu benutzen und muss Features erst freischalten, die er in anderen Fotosharing Apps standardmäßig hat. Genau das soll aber durch das Marketing als Besonderheit positiv aufgezeigt werden.
Die App ist zwar kostenlos aber wegen der In-App-Käufe in gewisser Weiße teurer als die konkurrierenden Apps. Auch die Features der App sind wie bereits erwähnt nicht unbedingt besser als die der Konkurrenten. Aus diesen Gründen werden wir mit der App eine Nischenstrategie fahren.

\section{Marketing-Ziel}

Das Ziel unseres Marketings ist es, innerhalb eines Jahres 10.000 aktive User weltweit zu generieren.

\section{Marketing-Strategie}

Auf den ersten Blick könnte sich fälschlicherweise der Eindruck ergeben, als hätte die Funktionsweise der App Nachteile für den Nutzer. Der User muss Bilder machen, um die App zu benutzen. Ausserdem muss er Features erst freischalten, die er in anderen Fotosharing-Apps standardmäßig hat. Genau diese Aspekte sollen aber durch das Marketing als Vorteile gepriesen werden.
Die App kann zwar kostenlos heruntergeladen werden, ist aber wegen der In-App-Käufe genau genommen teurer als die konkurrierenden Apps. Die App hat zwar besondere Alleinstellungsmerkmale, diese sind aber auf dem ersten Blick nicht einfach zu erkennen. Aus diesen Gründen werden wir mit der App eine Nischenstrategie fahren und beim Marketing einen besonderen Augenmerk darauf legen, dass diese Alleinstellungsmerkmale klar vermittelt werden.

\section{Markteinstiegsbarrieren}

Die größte Markteintrittsbarriere ist die Produktdifferenzierung. Die meisten Smartphone-User, die gerne ihre Fotos teilen, machen das bereits über andere Apps. Jedoch benutzen solche Smartphone-User gerne mehrere Fotosharing-Apps gleichzeitig, um mit den Fotos noch mehr Menschen zu erreichen. Dieser Punkt kann beim Markteintritt helfen, da sie mit unserer Fotosharing-App vor allem im Nischensegment noch mehr Leute erreichen können. Der Aspekt des Nischensegments bringt noch einen weiteren wichtigen Vorteil mit sich: User, die bewusst nur Fotos aus dem Nischenbereich suchen, finden mit unserer App erstmalig eine solche Gelegenheit. Anders formuliert könnte unsere bessere App aufgrund dieses Nischen-Alleinstellungsmerkmals besser passen als die Apps der Konkurrenz, was sie noch stärker dazu führen könnte, auf unsere App umzusteigen bzw. diese mitzubenutzen.


\section{Kundenbindung}

Unsere App möchte in erster Linie nicht nur User gewinnen, sondern diese auch langfristig an sich binden. Das ist darauf zurückzuführen, dass tendenziell eher die langfristigen User dazu neigen, In-App-Käufe zu tätigen. Aus diesem Grund ist die Kundenbindung sehr wichtig für unser Produkt. Um diese langfristige Bindung zu ermöglichen, soll im Marketing kommuniziert werden, dass es sich um eine Lifestyle-App handelt. Dadurch werden Benutzer angeworben, die an einer längeren Benutzung der App interessiert sind.
Des Weiteren soll den Benutzern unserer App und potentiellen neuen Usern gezeigt werden, dass die Macher der App sich um die Benutzer kümmern. Deshalb soll auf Bewertungen und Kommentaren im App-Store und auf dem Google Play Store immer geantwortet werden und auf Verbesserungsvorschläge eingegangen werden.

\section{Marke}

Die Marke der App zielt auf junge Leute ab, die sich der Natur oder dem Mysteriösen hingezogen fühlen. Im Anhang befindet sich der Creative Brief der App, der diese Marke beschreibt.
Der Name \textit{Imagical} geht aus dieser Marke hervor. Er setzt sich aus den englischen Wörtern \textit{image} und \textit{magical} zusammen. Der Name der App wurde bewusst aus englischen Wörtern zusammengesetzt, um für diese einfacher international werben zu können.
Mit dieser Marke ist es einfach zu kommunizieren, wieso die User gezwungen sind ein Bild hochzuladen. Die Idee ist hierbei, dass die User jeden Tag einen magischen Moment finden und festhalten sollen. Die App wirkt damit so, als würde sie die User nur dazu motivieren wollen auf magische Momente zu achten.
In der App wird die Marke mit der Metapher eines Himmels realisiert. Die Bilder die durchsucht werden können repräsentieren Wolken am Himmel. Das funktioniert, da die Bilder in der Übersicht verschwommen dargestellt werden. Die Punkte, die man in der App sammelt, sowie die Features die man mit diesen Punkten freischalten kann werden ebenfalls als Teile des Himmels dargestellt. Sie sollen in die Marke der App passen und auf keinen Fall so wirken, als seien sie ein aufgesetztes System um Geld zu verdienen. Deshalb müssen die freischaltbaren Features mit kryptischen Namen benannt werden und abstrakte Icons besitzen. Dadurch wirken sie konsistent mit der App in ihrer geheimnisvollen Art.
Im der Werbung wird auf die freischaltbaren Features und die In-App käufe kein Fokus gelegt, da sie von Außen eher als Nachteil gesehen werden. Bei genügend langer Benutzung wird der Vorteil der freischaltbaren Features jedoch jedem User klar.

\section{Werbemaßnahmen}

Als Werbemaßnahmen werden wir zum einen gesponsorte Beiträge auf Facebook, sowie Google Adwords Werbung verwenden.
In beiden Fällen können die Zielgruppen genau bestimmt werden. Das ist sehr hilfreich, da unser Produkt nur bestimmte Zielgruppen anspricht. Und weil wir ein digitales Produkt haben, macht es auch Sinn, dass die Werbung auch an diejenigen geht, die sich in der digitalen Welt bewegen.
Bei Google Adwords wird pro Klick gezahlt. Es muss deshalb darauf geachtet werden, dass unsere Werbebanner nicht besonders viele Klicks bekommen, sondern, dass jeder der dem Link folgt auch genau weiß, was beworben wird und ob es auf ihn passt. Ein Maximalwert für die Kosten kann angegeben werden. Diesen sollten wir bei etwa 10\euro am Tag festlegen und über 50 Tage laufen lassen. So können wir versuchen beständig neue Benutzer anzuwerben. Dadurch wird der Content der App auch beständig gefüllt, denn jeder neue Benutzer trägt als erstes etwas dazu bei. So kann verhindert werden, dass Benutzer die App wegen Mangels an neuen Bilder nach einer Weile uninteressant finden.
Dasselbe gilt für die Facebook-Beiträge: Hier kann der Preis der Werbung in ähnlicher Form angegeben werden. 10\euro pro Tag über 50 Tage sollten auch hier investiert werden.

\section{Mundpropaganda}

Diese Werbemaßnahmen sind nicht dafür gemacht, alle möglichen User zu erreichen. Die Marke der App und das Gemeinschaftsgefühl, das bei den Benutzern hervorgerufen wird, ist in einer Bannerwerbung schwer in seiner Ganzheit einzufangen. Dazu eignet sich Mundpropaganda um einiges besser. Die Werbung ist deshalb dazu ausgelegt eine genügend hohe Anzahl an Nutzern zu gewinnen, um den Content der App interessant zu machen. Die eigentliche Verbreitung geschieht dann durch Weiterempfehlung der begeisterten Benutzer.
Die beiden Werbekampangen sollten dabei gleichzeitig und nach Release der App starten. Die App muss bereits im App-Store sein, damit diejenigen, die auf die Werbung aufmerksam werden, diese auch sofort herunterladen können. Die App muss dann bereits mit Content gefüllt sein, welcher zuvor in der Betaphase der App entstanden ist. Laufen die beiden Kampagnen gleichzeitig und über einen längeren Zeitraum, so ist die Chance hoch, dass manche potenzielle Kunden die Werbung mehrmals auf unterschiedlichen Wegen zu sehen bekommt. Das führt dazu, dass diejenigen der Marke ein höheres Vertrauen schenken und eher dazu geneigt sind unser Produkt auszuprobieren.
