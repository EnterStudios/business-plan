\chapter{Marketing und Vertrieb}
Das Ziel unseres Marketings soll sein, durch die anstrengungen des Chief Marketing Officers bis in einem Jahr 10.000 aktive User zu generieren.
Die App hat auf den ersten Blick nur Nachteile für den User. Der User muss Bilder machen um die App zu benutzen und muss Features erst freischalten, die er in anderen Fotosharing Apps standardmäßig hat. Genau das soll aber durch das Marketing als besonderheit positiv aufgezeigt werden.
Die App ist zwar kostenlos aber wegen der In-App käufe in gewisser Weiße teurer als die konkurierenden Apps. Auch die Features der App sind wie bereits erwähnt nicht unbedingt besser als die der Konkurenten. Aus diesen Gründen werden wir mit der App eine Nischenstrategie fahren.

Die größte Markteinstiegsbarriere wird Produktdifferenzierung sein. Die meisten derer, die gerne Fotos machen um diese zu in einer App zu teilen, haben bereits eine App, die sie dafür benutzen. Dennoch können sie auch auf unser Produkt aufmerksam werden, da Benutzer in dieser Kategorie auch mehrere ähnliche apps gleichzeitig nutzen können, um mit den eigenen Bilder noch mehr Menschen zu erreichen.
Zudem wirkt die Nischenstrategie dieser Barriere entgegen. Bentzer von konkurierenden Apps steigen möglicherweiße auf unsere App um, da die Marke unserer App besser auf sie passt, als die der Konkurenz.

Von Nutzen sind nur Benutzer, die die App längere Zeit benutzen. Nur in diesem Fall beschliest ein Nutzer möglicherweise einen In-App kauf in der App zu tätigen. Aus diesem Grund ist die Kundenbindung sehr wichtig für unser Produkt. Im Marketing muss kommuniziert werden, dass es sich um eine Lifestyle App handelt. Dadurch werden Benutzer angeworben, die interesse an einer längeren Benutzung einer App interessiert sind.

Die Marke der App zielt auf junge Leute ab, die sich der Natur oder dem Mysteriösen hingezogen fühlen. Im Anhang befindet sich der Creative Brief der App, der diese Marke beschreibt.
Der Name Imagical geht aus dieser Marke hervor. Er setzt sich aus den Englischen Wörtern "image" und "magical" zusammen. Englisch ist der Name, um die App international vermarkten zu können.
Mit dieser Marke ist es einfach zu kommunizieren, wieso die User gezwungen sind ein Bild hochzuladen. Die Idee ist hierbei, dass die User jeden Tag einen magischen Moment finden und festhalten sollen. Die App wirkt damit so, als würde sie die User nur dazu motivieren wollen auf magische Momente zu achten.
In der App wird die Marke mit der Metapher eines Himmels realisiert. Die Bilder die durchsucht werden können räpresentieren Wolken am Himmel. Das funktioniert, da die Bilder in der Übersicht verschwommen dargestellt werden. Die Punkte, die man in der App sammelt, sowie die Features die man mit diesen Punkten freischalten kann werden ebenfalls als Teile des Himmels dargestellt. Sie sollen in die Marke der App passen und auf keinen Fall so wirken, als seien sie ein aufgesetzes System um Geld zu verdienen. Deshalb müssen die freischaltbaren Features mit Kryptischen Namen benannt werden und abstrakte Icons besitzen. Dadurch wirken sie konsistent mit der App in ihrer geheimnisvollen Art.
Im der Werbung wird auf die freischaltbaren Features und die In-App käufe kein Focus gelegt, da sie von Außen eher als Nachteil gesehen werden. Bei genügend langer Benutzung wird der Vorteil der freischaltbaren Features jedem User klar.

Als Werbemaßnahmen werden wir zum einen gesponsorte Beiträge auf Facebook, sowie Google Adwords Werbung. In beiden Fällen können die Zielgruppen genau bestimmt werden. Das ist sehr hilfreich, da unser Produkt nur bestimmte Menschen anspricht. Und weil wir ein digitales Produkt haben, macht es auch Sinn, dass die Werbung auch an diejenigen geht, die sich in der Digitalen Welt bewegen.
Bei Google Adwords wird pro Klick gezahlt. Es muss deshalb darauf geachtet werden, dass unsere Werbebanner nicht besonders viele Klicks bekommen, sondern, dass jeder der dem Link folgt auch genau weiß, was beworben wird und ob es auf ihn passt. Ein Maximalwert für die Kosten kann angegeben werden. Diesen sollten wir bei etwa 5€ am Tag festlegen und über 50 Tage laufen lassen. So können wir versuchen beständig neue Benutzer anzuwerben. Dadurch wird der Content der App auch beständig gefüllt, denn jeder neue Benutzer trägt als erstes etwas dazu bei. So kann verhindert werden, dass Benutzer die App wegen Mangels an neuen Bilder nach einer Weile uniteressant finden.
Das selbe gilt für die Facebook Beiträge. Hier kann der Preis der Werbung in ähnlicher Form angegeben werden. 5€ pro Tag über 50 Tage sollten auch hier investiert werden.
Die bieden Werbekampangen sollten dabei gleichzeitig und nach release der App starten. Die App muss bereits im App-Store sein, damit diejenigen die auf die Werbung aufmerksam werden die App auch sofort herunterladen können. Die App muss dann bereits mit Content gefüllt sein, der in der Betaphase der App entstanden ist. Laufen die beiden Kampangen gleichzeitig und über einen längeren Zeitraum ist die Chance hoch, dass manche Potenziellen Kunden die Werbung mehrmals auf unterschiedlichen Wegen zu sehen bekommt. Das führt dazu, dass diejenigen der Marke ein höheres Vertrauen schenken und eher dazu geneigt sind unser Produkt auszuprobieren.