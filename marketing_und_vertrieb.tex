\chapter{Marketing und Vertrieb}
Das Ziel unseres Marketings soll sein, durch die anstrengungen des Chief Marketing Officers bis in einem Jahr 10.000 aktive User zu generieren.
Die App hat auf den ersten Blick nur Nachteile für den User. Der User muss Bilder machen um die App zu benutzen und muss Features erst freischalten, die er in anderen Fotosharing Apps standardmäßig hat. Genau das soll aber durch das Marketing als besonderheit positiv aufgezeigt werden.
Die App ist zwar kostenlos aber wegen der In-App käufe in gewisser Weiße teurer als die konkurierenden Apps. Auch die Features der App sind wie bereits erwähnt nicht unbedingt besser als die der Konkurenten. Aus diesen Gründen werden wir mit der App eine Nischenstrategie fahren.

Die Marke der App zielt auf junge Leute ab, die sich der Natur oder dem Mysteriösen hingezogen fühlen. Im Anhang befindet sich der Creative Brief der App, der diese Marke beschreibt.
Der Name Imagical geht aus dieser Marke hervor. Er setzt sich aus den Englischen Wörtern "image" und "magical" zusammen. Englisch ist der Name, um die App international vermarkten zu können.
Mit dieser Marke ist es einfach zu kommunizieren, wieso die User gezwungen sind ein Bild hochzuladen. Die Idee ist hierbei, dass die User jeden Tag einen magischen Moment finden und festhalten sollen. Die App wirkt damit so, als würde sie die User nur dazu motivieren wollen auf magische Momente zu achten.
In der App wird die Marke mit der Metapher eines Himmels realisiert. Die Bilder die durchsucht werden können räpresentieren Wolken am Himmel. Das funktioniert, da die Bilder in der Übersicht verschwommen dargestellt werden. Die Punkte, die man in der App sammelt, sowie die Features die man mit diesen Punkten freischalten kann werden ebenfalls als Teile des Himmels dargestellt. Sie sollen in die Marke der App passen und auf keinen Fall so wirken, als seien sie ein aufgesetzes System um Geld zu verdienen. Deshalb müssen die freischaltbaren Features mit Kryptischen Namen benannt werden und abstrakte Icons besitzen. Dadurch wirken sie konsistent mit der App in ihrer geheimnisvollen Art.
Im der Werbung wird auf die freischaltbaren Features und die In-App käufe kein Focus gelegt, da sie von Außen eher als Nachteil gesehen werden. Bei genug aktiven Usern wird der Vorteil der freischaltbaren Features jedem User klar.