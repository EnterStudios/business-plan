\chapter{Produkt und Dienstleistung}

Fotosharing App
Um in die App zu kommen muss man einmal pro Tag ein Foto machen und hochladen
-> Öffnet man die App und hat an diesem Tag noch kein Foto hochgeladen, muss man erst eines machen und hochladen bevor man die Bilder der Anderen sehen, oder sonst irgendwas machen kann.
Man kann keine Fotos aus der Galerie hochladen, nur welche, die man mit der App aufnimmt

In der App kann man die Bilder aller anderen User durchstöbern und ansehen. Das ist die Hauptfunktion der App - Deshalb benutzen die User die App im Grunde

Schaut man mit der App ein Bild in der Detailansicht an, muss man dieses Bewerten, um zurück zur Bilderübersicht zu kommen. Dadurch werden Alle Bilder von vielen Bewertet und schlechte und gute Bilder können einfach identifiziert werden.

Alle normalen Features, die über die oben genannten Grundfeatures hinaus gehen, (z.B. Kommentare schreiben, Filter für Fotos benutzen, username angeben, ...) müssen erst für Punkte freigeschaltet werden.
Punkte bekommen User, wenn ihre Bilder gut bewertet werden. Dadurch haben sie den Ansporn nur gute Bilder zu machen.
Diese Punkte können auch durch In-App käufe erworben werden.


-----

Ein Geschäftsvorhaben gründet sich auf eine Produkt- oder Dienstleistungsidee.
Das Wichtigste dabei ist, den Nutzen f r die zukünftigen Kunden herauszustellen.
Wie unterscheidet sich Ihr Produkt von den Produkten
Der Wettbewerber?
Kurze Darstellung des Stands der Produktentwicklung und der erforderlichen weiteren Schritte.

Dimension des Kundennutzens: Zeit, Kosten und Qualität. Ein Produkt braucht mindestens 1 oder 2 davon.

\begin{itemize}
\item Was ist Ihr Produkt, Ihre Dienstleistung?
\item Worin besteht die Innovation Ihrer Idee? Merkmale!
\item Wie sieht der aktuelle Stand der Technik aus?
\item Planen Sie weitere Varianten, zusätzliche Produkte, Dienstleistungen?
\item Welchen Kundennutzen bietet Ihr Produkt /Ihre Dienstleistung?
\item Was ist der relevante Kundennutzen?
\item Welche Annahmen legen Sie Ihren Quantifizierungen zugrunde?
\item Welche Zielkundengruppen und/oder welche Endkundengruppen sprechen Sie an?
\item Welche Versionen Ihres Produkts/Ihrer Dienstleistung sind für welche Kundengruppen und Anwendungsarten gedacht?
\item Wie sieht Ihr Service- und Wartungsangebot aus?
\item Welche Produkt-/Dienstleistungsgarantien geben Sie?
\item Welche Konkurrenzprodukte zu Ihrem Produkt existieren bereits oder sind in Entwicklung und wie unterscheiden sich diese von Ihrem Produkt?
\item Aus welchen Gründen ist Ihr Produkt/Ihre Dienstleistung (oder vergleichbare Konkurrenzprodukte) noch nicht auf dem Markt?
\item Welche Voraussetzungen sind für die Entwicklung und Herstellung erforderlich und erfüllen Sie diese bereits? – Stadium der Entwicklung.
\item Welche Entwicklungsziele müssen erreicht werden? 
\item Welche Entwicklungsschritte planen Sie?
\end{itemize}