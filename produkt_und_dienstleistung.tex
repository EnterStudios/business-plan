\chapter{Das Produkt}


\section{Grundidee}

Bei unserem Produkt \textit{Imagical} handelt es sich um eine Fotosharing-App, die bekannten Apps wie Instagram, Twitter oder Pinterest ähnelt. Gleichzeitig gibt es jedoch signifikante Unterschiede, die unsere App von den existierenden Apps abgrenzt:
So muss ein Benutzer beim Start der App \textit{Imagical} täglich ein Foto erstellen und veröffentlichen, bevor Fotos anderer Benutzer angeschaut oder weitere Aktionen durchgeführt werden können. Hierdurch wird sichergestellt, dass stets genügend Bildmaterial vorhanden ist, um der Community genügend Abwechslung bieten zu können. Durch diese Maßnahme kann auch bereits mit einer geringen Benutzeranzahl eine größere Menge an Content als bei anderen Sharing-Apps gewährleistet werden.

Zudem gibt die App vor, dass nur durch die App selbst aufgenommene Bilder hochgeladen werden können; zuvor aufgenommene Bilder aus der Bildergalerie werden somit nicht unterstützt.
Hierdurch wird erzielt, dass ein Benutzer das selbst Bild nicht mehrmals hochladen kann, wodurch andere Benutzer ein und dasselbe Bild nicht öfters zu sehen bekommen.
Ein weiterer Vorteil dieser Regelung ist, dass die aufgenommenen Bilder dadurch eine sehr authentische und exklusive Wirkung beim Benutzer erzielen. Zudem können somit Bilder, die in der App hochgeladen werden, nicht in anderen Apps wiedergefunden werden.

\section{Besonderheiten}

In der App kann man die Bilder aller anderen User durchstöbern und ansehen. Der hauptsächliche Kundennutzen der App gegenüber den Konkurenten, ist die Art an Bildern, die gezeigt werden. Man folgt nicht berühmten Persöhnlichkeiten oder bekommt Bilder, die im Internet gefunden wurden, stattdessen sieht man nur die beliebtesten Bilder, die andere Benutzer mit der App selbst gemacht haben. Da jeder, der die App benutzt, auch selbst Bilder hochlädt, entsteht automatisch ein Gefühl der Zugehörigkeit. Das trägt unheimlich der Kundenbindung bei, was vor allem für das Geschäftsmodell eine sehr große Rolle spielt.

Die Bilder werden dabei in der Übersicht zuerst verschwommen angezeigt.
Dieses Feature unterscheidet \textit{Imagical} von andern Apps und bietet zudem gleich mehrere Vorteile: Aus technischen Gründen ist es gut, die Bilder in der Übersicht mit niedriger Auflösung zu laden. Durch die verschwommene Anzeige können die Bilder als extrem verkleinerte Versionen heruntergeladen werden, wodurch die App auch bei einer langsamen Internetverbindung gut benutzbar ist und die Nutzer sich nicht über große Datennutzung beschweren müssen. Es gibt auch eine gewisse Befriedung als Benutzer, wenn man ein Bild erst verschwommen sieht und sich unterbewusst fragt, was dort wohl dargestellt wird. Der Benutzer kann dann durch einen Klick auf die verschwommene Vorschau das Bild in voller Pracht (Detailansicht) zu sehen bekommen.

Noch ein Grund für die verschwommene Vorschau ist, dass damit jedes angesehene Bild bewertet wird, was mit dem Prinzip der Bewertung in der App zusammenhängt:
Schaut ein Benutzer ein Bild in der Detailansicht an, muss er dieses Bewerten, um zurück zur Bilderübersicht zu kommen. Dadurch werden alle Bilder von vielen bewertet, wodurch gute und schlechte Bilder einfach unterschieden werden können. Zudem kann die Auswahl der Bilder, die jedem Benutzer angezeigt werden, personalisiert werden. Bei vielen Benutzern ist die Anzahl der neuen Bilder pro Tag hoch. Durch die Personalisierung können sich dann unterschiedliche Communities innerhalb der App bilden, die jeweils eine etwas andere Art von Bildern bevorzugen.

\section{Der spielerische Aspekt}

Das Bewerten aller Bilder ist zudem wichtig, weil ein Benutzer Punkte für seine Bilder bekommt, je nachdem wie gut diese bewertet wurden. Alle normalen Features, die über die oben genannten Grundfeatures hinaus gehen (z.B. Kommentare schreiben, Filter für Fotos benutzen, Benutzername angeben, ...) können erst durch zuvor erreichte Bewertungspunkte freigeschaltet werden. Dadurch haben die Benutzer den Ansporn nur gute Bilder zu machen.
Das Freischalten von Features mittels Punkten in einer App kennen Benutzer wahrscheinlich eher von Spielen.
Das eingeschränkte Feature-Set von \textit{Imagical} könnte vom Benutzer am Anfang als Nachteil empfunden werden, da ihm zu Beginn nicht die volle Funktionalität zur Verfügung stellt.
Jedoch ist das Prinzip des Fortschritts sehr spaßig. Die Benutzer haben einen Ansporn, sich weiterhin aktiv mit der App zu beschäftigen.

\section{Geschäftsmodell}

Die App wird kostenlos angeboten. Das Geschäftsmodell der App basiert auf In-App-Käufen. Die Punkte, die zur Freischaltung von Features benötigt werden, können so in der App gekauft und bezahlt werden. Wegen des starken Gemeinschaftsgefühls in der App und der Involvierung jedes Benutzers, ist es den den meisten wichtig, alle Möglichkeiten der App nutzen zu können. Aus diesem Grund wird ihnen ein entsprechender In-App-Kauf das Geld wert sein.

\section{Stand der Entwicklung}

Die App ist am Ende der Konzeptionsphase und muss nun nur noch umgesetzt werden. Aufgrund der freischaltbaren Features können jedoch auch nach dem Release der App immer weitere Funktionalitäten hinzugefügt werden. Damit wird dann gleichzeitig der Drang nach genügend Punkten erhöht, um diese benutzen zu können.
