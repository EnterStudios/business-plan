\chapter{Produkt und Dienstleistung}


\section{Grundidee}

Bei unserem Produkt handelt es sich um eine Fotosharing App. Damit ähnelt sie bekannten Apps wie Instagram und anderen großen Apps. Ein Markt für diese Art Apps ist existiert also. Einige wichtige Unterschiede gibt es, die unsere App von den existierenden Apps unterscheidet.
Um in die App zu kommen muss man einmal pro Tag ein Foto machen und hochladen. Öffnet man die App und hat an diesem Tag noch kein Foto hochgeladen, muss man erst eines machen und hochladen, bevor man die Bilder der Anderen sehen, oder sonst irgendwas machen kann. Das führt dazu, dass die App davor geschützt ist, aus mangel an Content uninteressant zu sein. Zum einen, weil ein neuer Benutzer zuerst etwas zur App beiträgt, bevor er überhaupt sieht, wie belebt die App ist. Zum anderen wird so auch mit wenigen Benutzern um einiges mehr Content generiert als bei anderen Sharing-Apps.
Man kann keine Fotos aus der Galerie hochladen, nur welche, die man mit der App aufnimmt. Dadurch wird verhindert, dass Benutzer das selbe Bild öffters zu sehen bekommen kann und dadurch gelangweilt werden könnte. Außerdem wirken die Bilder in der App dadurch authentisch und exklusiv. Bilder die in unserer App zu sehen sind, können nicht in anderen Apps gefunden werden.

\section{Besonderheiten}

In der App kann man die Bilder aller anderen User durchstöbern und ansehen. Der hauptsächliche Kundennutzen der App gegenüber den Konkurenten, ist die Art an Bildern, die gezeigt werden. Man folgt nicht berühmten Persöhnlichkeiten oder bekommt Bilder, die im Internet gefunden wurden. Statt dessen sieht man nur die beliebtesten Bilder, die Menschen mit der App selbst gemacht haben. Da jeder der die App benutzt auch selbst Bilder hochlädt, entsteht automatisch ein Gefühl der Zugehörigkeit. Das trägt unheimlich der Kundenbindung zu, was für das Geschäftsmodell sehr wichtig ist.
Die Bilder werden dabei in der Übersicht zuerst verschwommen angezeigt. Auch das ist ein Unterschied zu ähnlichen Apps, der mehrere Gründe hat. Aus technischen Gründen ist es gut, die Bilder in der Übersicht mit niedriger Auflösung zu laden. Durch das verschwommene Anzeigen können die Bilder als extrem verkleinerte Versionen heruntergeladen werden. Der Hauptvorteil davon ist, dass die App auch bei schlechtem Internet benutzbar ist und die Nutzer sich nicht über große Datennutzung beschweren müssen. Es gibt auch eine gewisse Befriedung als Benutzer, wenn man ein Bild erst verschwommen sieht, sich unterbewusst fragt, was dort dargestellt ist und dann erst das Bild in voller Pracht zu sehen bekommt. Noch ein Grund für die verschwommene Vorschau ist, dass damit jedes angesehene Bild bewertet wird, was mit dem Prinzip der Bewertung in der App zusammenhängt.
Schaut man mit der App ein Bild in der Detailansicht an, muss man dieses Bewerten, um zurück zur Bilderübersicht zu kommen. Dadurch werden alle Bilder von vielen bewertet und schlechte und gute Bilder können einfach unterschieden werden. Zudem kann die Auswahl der Bilder, die jedem Benutzer angezeigt werden, personalisiert werden. Bei vielen Benutzern ist die Anzahl der neuen Bilder pro Tag hoch. Durch die Personalisierung können sich dann unterschiedliche Communities innerhalb der App bilden, die jeweils eine etwas andere Art von Bildern bevorzugen.

\section{Der spielerische Aspekt}

Das Bewerten aller Bilder ist zudem wichtig, weil ein Benutzer Punkte für seine Bilder bekommt, je nachdem wie gut diese bewertet werden. Alle normalen Features, die über die oben genannten Grundfeatures hinaus gehen (z.B. Kommentare schreiben, Filter für Fotos benutzen, username angeben, ...) müssen erst für Punkte freigeschaltet werden. Dadurch haben die Benutzer den Ansporn nur gute Bilder zu machen.
Das freischalten von dingen mittels Punkten in einer App kennen Benutzer eher von Spielen. Bei einer anderen App könnte man zuerst meinen, dass es ein Nachteil wäre, dass man als Benutzer am Anfang nicht alles machen kann, was andere können. Jedoch ist das Prinzip des Fortschritts sehr spaßig. Die Benutzer haben einen Ansporn, sich aktiv mit der App zu beschäftigen.

\section{Geschäftsmodell}

Die App wird kostenlos angeboten. Das Geschäftsmodell der App basiert auf In-App käufen. Die Punkte die zur Freischaltung von Features benötigt werden können in der App gekauft werden. Wegen des starken Gefühls der Gemeinschaft in der App und der involvierung jedes Benutzers, ist es den den meisten wichtig, alle Möglichkeiten der App nutzen zu können. Aus diesem Grund wird ihnen ein In-App kauf ihr Geld wert sein.

\section{Stand der Entwicklung}

Die App ist am Ende der Konzeptionsphase und muss nun nur noch umgesetzt werden. Aufgrund der freischaltbaren Features können jedoch auch nach dem Release der App immer weitere Funktionalitäten hinzugefügt werden. Damit wird dann gleichzeitig der Drang nach genügend Punkten erhöht, um diese benutzen zu können.
